\chapter{Testing Models of Signal Transmission Along Nerve Axons}
\thispagestyle{fancy}
\fancyhead[RE,LO]{Experiment \thechapter}
%
Nerve impulses travel in our bodies as electrical signals. 
Whether its seeing or hearing something, controlling a muscle, or just thinking, the transmission process along a nerve cell, or neuron, is the same: a sufficient stimulus received by the cell body (soma) initiates a change in the potential difference across the membrane, or action potential, which travels along the axon to be transferred through a synapse to the other neurons or muscle cells. 
\par
A single axon can be a meter or more long, like those connecting our toes to our spinal cords, so action potentials must travel a long way.
You may have learned in a biology course that for an action potential to be transmitted from one end of an axon to another, it must be regenerated repeatedly along its length.
We call this active transport and it is accomplished by voltage-gated ion channels, as discussed in the appendix to this lab.
Why is active transport necessary?
Couldn't we simply apply a potential difference to the end of a nerve axon and expect that signal to travel along the axon to its final destination, the spinal cord and then the brain?
This alternate type of signal transmission is called passive transport. 
In this two week lab sequence, transport is a feasible method of signal transmission. 
We will also be considering some adaptations to the simple cylinder model of a nerve axon (see below) that can increase the speed of signal transfer – surely a faster response is advantageous to survival in a species!

\subsection*{Simple cylinder model of a nerve axon: structure and properties}
The structure of the axon is shown schematically in figure 3 below.
for electrical purposes, an axon is basically a long, thin cylinder of membrane filled with a fluid called axoplasm.
when no action potential is traveling, the potential difference across the membrane, from the outside of the axon to the inside, is about -70 mV, the resting potential difference.
In part, this is because the concentration of of sodium atoms (Na$^{+}$) is much higher outside the axon than inside.
\par
the axoplasm, like all fluids in biological systems, has mobile ions dissolved in it, giving it a moderate conductivity (many orders of magnitude lower than copper or other metals we can think of as ideal conductors).
the fluid outside the membrane, the extra-cellular fluid, is very similar to the axoplasm and thus has approximately the same conductivity.
\par 
The lipid bilayer forming the membrane is electrically insulating, with extremely low conductivity, but many different kinds of channel proteins cross the membrane.
These channels only allow specific ions to pass under particular conditions,
the channels increase the conductivity of the membrane as a whole, so that although the membrane conductivity is much lower than that of the axoplasm, current does not pass through the membrane.

\section*{Part 1: Modeling Passive Transport}
Our goal is to understand how a potential difference applied across the membrane at one end of the axon spreads along the axon if it is not being regenerated as it travels down the axon.
thus for the purposes of our model we'll assume there is a battery at the left end of the axon segment (see the figure), holding the potential difference from the outside to the inside at $x=0$ to a positive value we'll call $V_{0}$.
\par 
Suppose there was just a single ion channel (e.g., a `potassium leak channel') crossing through the membrane, at the right end of the segemnt in the figure. 
Draw a loop onto the diagram showing the path of current flow from the positive side of the battery (inside) to the negative side of the battery (outside), and then draw a simple circuit that corresponds to this pattern of current flow.
The axoplasm, an ion channel, and the extra-cellular fluid should each be represented separately in your simple circuit.
\par 
In fact, the resistance of the current path through the extra-cellular is much less than the resistance of either the axoplasm or the ion channel through the membrane, so the extra-cellular fluid can be represented as just a conducting wire in the simple circuit.
(If you didn't initially do so, update your simple circuit accordingly.) 
Let's think about why the extracellular fluid will have such low resistance.
\par 
Resistance $R$ can be determined from the length $L$, the cross-sectional area $A$, and the resistivity $\rho$ according to $R = \frac{\rho L}{A}$.
The resistivities of the axoplasm and the extracellular fluid are very similar.
Considering how the current flows through both the axoplasm and the extra-cellular fluid, explain why the resistance of the current path through the extra-cellular fluid is much smaller than the resistance of the current path through the exoplasm - thus we cannot treat the extra-cellular fluid as just a conducting wire.
\par 
A long axon can be modeled as a chain of many short segments like the segments like the segment described above.
The dashed lines in figure 4 visually divide the axon into five such segments.
sketch the path of current flow onto the diagram and then design a circuit model for it.
Check your model with your instructor.
\par 
To complete the model, we need values for the resistances.
There are actually many ion channels passing through the membrane, so our circuit model needs to be constructed using the resistance of a segment of membrane rather  than a single ion channel.
Each segment of membrane consists of many ion channels in parallel, so the overall resistance of the membrane segment is the equivalent resistance of all of those ion channels. 
The average resistivity of the membrane can be measured and used to find the resistance of a segment of membrane.
\par 
Use the relationship between resistance, resistivity, length, and cross-sectional area to estimate values for the resistance of a membrane segment $R_{mem}$ and an axoplasm segment $R_{axon}$ using the following order-of-magnitude values:
\begin{itemize}
\item the diameter of the axon $\sim$ 10 $\mu$m
\item the membrane thickness $\sim$ 10 nm
\item the resistivity of the axoplasm $\sim$ 1 $\Omega$-m
\item the average resistivity of the membrane $\sim  10^{8} \; \Omega$-m
\item the segment length $\sim$ 1 mm
\end{itemize}
Hint: To determine the cross-sectional area of a membrane segment, think of unrolling the membrane from the axon.
Also, it turns out that only the ratio $R_{mem}/R_{axon}$ affects how far the potential difference will spread.

\section*{Part 2: Qualitative Analysis of Your Model}
With the model you have created, we hope to understand how the potential difference across the membrane changes as we examine different distance along the nerve axon.
In other words, how is the voltage across the membrane at a distance $x$ related to the initial voltage $V_{0}$?
Before we collect data and perform a quantitative analysis of your model, let us think qualitatively about some of the features of your model of passive transport.
\par 
We want to understand how the voltage across the membrane depends on the distance along the axon.
Each segment in you multi-segment model circuit 