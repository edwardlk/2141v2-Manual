\chapter{Exploring Light and Lenses}
\thispagestyle{fancy}
\fancyhead[RE,LO]{Experiment \thechapter}
%
Optics — the sub-field of physics focusing on the study of light — is important to many areas of biology, including vision, ecology, botany, neurobiology and molecular biology. 
The use of optics in biology has evolved from the simple light microscope used by Darwin to the complex, live-cell, high resolution microscopes used in current cutting-edge research. 
In this two-week lab, we will be exploring the behavior of light and lenses. 
Microscopes usually employ at least two lenses together—but, as we are only beginning to learn about light, we will be exploring single-lens systems. 
\par 
Your task is two-fold: 1) Design a way of determining the focal lengths, f, for five converging (bi-convex) lenses; and then, 2) Design an experiment to explore how the focal length, f, of a lens interacts with the image distance, i, and object distance, o, when the total distance, from object to image, is a fixed length, L. 
The image distance, i, is the distance from the location of the image to the center of the lens. 
The object distance1, o, is the distance from the location of the object to the center of the lens. 
For most microscopes, the object location and image location cannot be changed (they are a fixed distance, L apart), so it is the type of lens and the location of the lens that are changed to achieve different magnifications and to focus the image. 
Ultimately, you should be able to construct a mathematical model of the interaction of focal length and image/object distance that is supported by your data. 
Think carefully about your experimental designs and how they affect your sources (and values!) of uncertainty.
\par
\section*{About the Equipment:}
Each group has access to:
\begin{itemize}
\item a light source (your object) with pole clamp,
\item five converging (bi-convex) lenses of differing focal lengths,
\item one diverging (bi-concave) lens,
\item a lens holder with pole clamp,
\item a long optical rail (pole) with table clamp,
\item small LED light boxes emitting different colors of light,
\item a light box with a plano-convex lens,
\item a meter stick and a ruler, and
\item a simple light microscope à la Darwin.
\end{itemize}
Be sure that your optical elements, such as lenses, are carefully aligned with the vertical axis of your equipment. 
If you need help with this, or are not sure why this is important, please ask a TA/LA. 
Additionally, please be gentle with the equipment, especially the glass lenses, and try to avoid getting finger-prints, skin oils, or dirt on the lenses. Do not over-tighten clamps — if you are unsure, ask for help.
\par
\section*{For your report:}
The lab write-up should include careful discussions of (a) your methods/protocols for finding the focal lengths and investigating the f/i/o relationships, (b) your data and analysis of the data and the sources of error, (c) your mathematical model for the interaction of focal length and image/object distance, and (d) your comparison of your work with the work of other groups, as well as your critique of your experiment and conclusions. 
If you want to include hand-drawn elements (such as ray diagrams), please do so.
Please also consider whether the system you have investigated in your experiment is comparable to the simple light microscope that has been provided to you. 
In what ways is it similar/different?
What is 'adjustable' in this simple light microscope?
How could you go about investigating this new optical system?